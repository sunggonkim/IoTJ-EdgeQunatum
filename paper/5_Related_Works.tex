\section{Related Work}
\label{sec:related}

\subsection{GPU-Accelerated Quantum Simulation}

\textbf{cuQuantum}~\cite{cuquantum} provides NVIDIA's official GPU-accelerated quantum simulation primitives. cuStateVec achieves high throughput via $O(2^n)$ parallel gate operations but assumes in-memory state vectors, limiting scalability to GPU VRAM capacity (typically 30-33 qubits on 80GB A100).

\textbf{ScaleQsim}~\cite{10.1145/3771577} extends cuQuantum with distributed multi-GPU execution using bitwise index partitioning. Evaluated on 512 A100 GPUs, it achieves 42-qubit simulation with 77.4$\times$ speedup over single-GPU baselines. However, it requires HPC infrastructure costing millions of dollars.

\textbf{HyQuas}~\cite{zhang2021hyquas} employs precompiled gate kernels for reduced overhead. However, kernel compilation assumes fixed qubit configurations, limiting flexibility when memory constraints prevent the planned configuration.

\textbf{Atlas}~\cite{xu2024atlas} uses static execution planning with gate fusion optimization. Its reliance on fixed qubit-GPU mappings limits flexibility when memory constraints prevent the planned configuration.

\subsection{Storage-Extended Simulation}

\textbf{SnuQS}~\cite{park2022snuqs} extends simulation capacity via SSD offloading, demonstrating 42-qubit execution on CPU clusters. However, its CPU-centric design achieves 16+ hours for 42 qubits, compared to seconds on GPU-accelerated systems.

\textbf{BMQSim}~\cite{zhang2025bmqsim} employs GPU-side compression to reduce memory footprint. While effective, compression kernels compete with gate execution for GPU resources, potentially reducing throughput.

\subsection{Tensor Network Methods}

Tensor network simulators like \textbf{quimb}~\cite{gray2018quimb} decompose quantum states using tensor factorization. While memory-efficient for shallow circuits, deep circuits with high entanglement require exponential bond dimensions, negating the advantage.

\subsection{Edge and IoT Quantum}

Prior work on edge quantum computing has focused on:
\begin{itemize}
    \item \textbf{Quantum Key Distribution (QKD)}: Hardware protocols for secure communication
    \item \textbf{Post-Quantum Cryptography (PQC)}: Algorithm implementations resistant to quantum attacks
    \item \textbf{Hybrid Quantum-Classical}: Cloud-based quantum with edge preprocessing
\end{itemize}

PennyLane~\cite{pennylane} supports hybrid quantum-classical workflows but lacks edge-specific optimizations. To our knowledge, \EdgeQuantum is the first framework enabling large-scale (37-qubit) quantum simulation on sub-10W ARM-based edge devices.

\subsection{Positioning of EdgeQuantum}

Table~\ref{tab:positioning} compares \EdgeQuantum with key prior work.

\begin{table}[H]
\centering
\caption{Comparison with Prior Quantum Simulators}
\small
\begin{tabular}{lcccr}
\toprule
\textbf{System} & \textbf{Edge} & \textbf{Compress} & \textbf{Tiered} & \textbf{Max Q} \\
\midrule
cuQuantum & \xmark & \xmark & \xmark & 33 \\
ScaleQsim & \xmark & \xmark & \cmark & 42 \\
SnuQS & \xmark & \xmark & \cmark & 42 \\
BMQSim & \xmark & \cmark & \xmark & 36 \\
\midrule
\textbf{EdgeQuantum} & \cmark & \cmark & \cmark & \textbf{37} \\
\bottomrule
\end{tabular}
\label{tab:positioning}
\end{table}

\EdgeQuantum combines GPU acceleration (cuQuantum), storage offloading (SnuQS), and compression (BMQSim) into a unified framework optimized for resource-constrained edge devices.
